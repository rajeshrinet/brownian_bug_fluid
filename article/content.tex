
\section*{Introduction}
This article is a reproduction of \cite{young_reproductive_2001}.

\section*{Methods}

\subsection*{Analytical solution of G}
\subsubsection*{Derivation of G(r,t)}

Finding back Eq. (2) in the original paper?

\begin{equation}
\frac{\partial G}{\partial t}=2Dr^{1-d}\frac{\partial}{\partial r}\left(r^{d-1}\frac{\partial G}{\partial r}\right)+2(\lambda-\mu)G+\gamma r^{1-d}\frac{\partial}{\partial r}\left(r^{d+1}\frac{\partial G}{\partial r}\right)+2\lambda C\delta(\boldsymbol{r})\label{eq:eq_2_Young_total}
\end{equation}

We will focus on the case $d=2$ and $\lambda=\mu$, which means Eq.
(\ref{eq:eq_2_Young_total}) can be reduced to

\begin{equation}
\frac{\partial G}{\partial t}=\frac{2D}{r}\frac{\partial}{\partial r}\left(r\frac{\partial G}{\partial r}\right)+\frac{\gamma}{r}\frac{\partial}{\partial r}\left(r^{3}\frac{\partial G}{\partial r}\right)+2\lambda C\delta(\boldsymbol{r})\label{eq:eq_2_Young_reduced}
\end{equation}


\subsubsection*{Analytical solution with advection}

In the presence of advection ($\gamma\neq0$), a steady-state solution
can be found. 

\begin{align}
 & 0 & = & \frac{2D}{r}\frac{\partial}{\partial r}\left(r\frac{\partial G}{\partial r}\right)+\frac{\gamma}{r}\frac{\partial}{\partial r}\left(r^{3}\frac{\partial G}{\partial r}\right)+2\lambda C\delta(\boldsymbol{r})\nonumber \\
\Leftrightarrow & 0 & = & 2\pi r\left(\frac{2D}{r}\frac{\partial}{\partial r}\left(r\frac{\partial G}{\partial r}\right)+\frac{\gamma}{r}\frac{\partial}{\partial r}\left(r^{3}\frac{\partial G}{\partial r}\right)+2\lambda C\delta(\boldsymbol{r})\right)\nonumber \\
\Leftrightarrow & 0 & = & 2\pi\left(2D\frac{\partial}{\partial r}\left(r\frac{\partial G}{\partial r}\right)+\gamma\frac{\partial}{\partial r}\left(r^{3}\frac{\partial G}{\partial r}\right)\right)+2\pi r2\lambda C\delta(\boldsymbol{r})\label{eq:steady_state}
\end{align}

We can then integrate Eq. (\ref{eq:eq_2_Young_reduced}) over a small
area centered on a particle, with radius $\rho$. Let us first note
that

\begin{eqnarray}
\int_{\mathbb{R}^{2}}\delta(\boldsymbol{r})d^{2}\boldsymbol{r} & = & 1\nonumber \\
\Leftrightarrow\int_{0}^{2\pi}\int_{0}^{\rho}\delta(\boldsymbol{r'})r'dr'd\theta & = & 1\nonumber \\
\Leftrightarrow2\pi\int_{0}^{\rho}\delta(\boldsymbol{r'})r'dr' & = & 1\label{eq:delta_integration}
\end{eqnarray}

Using Eq. (\ref{eq:steady_state}) and (\ref{eq:delta_integration}),
we can integrate between 0 and $\rho$, 

\begin{align}
 & 0 & = & 2\pi\left(2D\rho\frac{\partial G}{\partial r}+\gamma\rho^{3}\frac{\partial G}{\partial r}\right)+2\lambda C\nonumber \\
\Leftrightarrow & \frac{\partial G}{\partial r} & = & -\frac{1}{2\pi}\frac{2\lambda C}{2D\rho+\gamma\rho^{3}}\label{eq:deriv_G_r}
\end{align}

We can integrate between $\rho$ and $\infty$, knowing that $G(\infty)=C^{2}.$

\begin{align}
 & C^{2}-G(\rho) & = & -\frac{1}{2\pi}{\displaystyle \int_{\rho}^{\infty}}\frac{2\lambda C}{2Dr+\gamma r^{3}}dr\label{eq:deriv_G_r_int1}
\end{align}

Using the variable change $u=2Dr+\gamma r^{3}$, the integral is equivalent
to $\int\frac{u'}{u}du$ 

\begin{align}
 & C^{2}-G(\rho) & = & -\frac{\lambda C}{2\pi}\frac{1}{4D}[\log(\gamma)-\log(\frac{2D}{r^{2}}+\gamma)]\label{eq:deriv_G_rint2}\\
\Leftrightarrow & G(\rho) & = & \frac{\lambda C}{8\pi D}\log\left(\frac{2D+\gamma r^{2}}{\gamma r^{2}}\right)+C^{2}\label{eq:G_rho}
\end{align}

Finally, the pair correlation function $g=G/C^{2}$ is defined as

\begin{equation}
g=\frac{\lambda}{8\pi DC}\log\left(\frac{2D+\gamma r^{2}}{\gamma r^{2}}\right)+1
\end{equation}


\subsubsection*{Analytical solution without advection}

When $U=0$, $\gamma=0$ and there is no steady solution. We can get
back to Eq. (\ref{eq:eq_2_Young_reduced}). 

\begin{equation}
\frac{\partial G}{\partial t}=\frac{2D}{r}\frac{\partial}{\partial r}\left(r\frac{\partial G}{\partial r}\right)+2\lambda C\delta(\boldsymbol{r})\label{eq:g_without_advection}
\end{equation}

Assuming an isotropic environment, this means

\begin{equation}
\frac{\partial G}{\partial t}-2D\Delta G=2\lambda C\delta(\boldsymbol{r})
\end{equation}

where $\Delta=\nabla^{2}$ is the Laplacian operator. 

We therefore have 

\begin{equation}
\mathcal{L}G(\boldsymbol{r},t)=2\lambda C\delta(\boldsymbol{r})\label{eq:LG_lambda}
\end{equation}

where $\mathcal{L}$ is the linear differential operator $\partial_{t}-2D\Delta$. 

Let's use the Green's function H, defined with $\mathcal{L}H=\delta(\boldsymbol{r},t)=\delta(\boldsymbol{r})\delta(t)$. 

By definition, we know that $G(y)=\int H(y,s)2\lambda C\delta(s)ds$
(where $y=(\boldsymbol{r},t)$) is a solution to Eq.(\ref{eq:LG_lambda}).

\begin{align}
 & G(\boldsymbol{r},t) & = & 2\lambda C\int_{\mathbb{R}^{2}}\int_{0}^{t}H(\boldsymbol{r}-\boldsymbol{r'},t')\delta(\boldsymbol{r'})d\boldsymbol{r}'dt'\nonumber \\
\Leftrightarrow &  & = & 2\lambda C\int_{0}^{t}H(\boldsymbol{r},t')dt'\label{eq:g_int_H}
\end{align}

Let's substitute Eq.(\ref{eq:g_int_H}) in Eq. (\ref{eq:g_without_advection}):

\begin{alignat*}{2}
 & \frac{\partial}{\partial t}\left(2\lambda C\int_{0}^{t}H(\boldsymbol{r},t')dt'\right) & = & 2D2\lambda C\Delta\int_{0}^{t}H(\boldsymbol{r},t')dt'+2\lambda C\delta(\boldsymbol{r})\\
\Leftrightarrow & \int_{0}^{t}\left(\frac{\partial H(\boldsymbol{r},t')}{\partial t'}-2D\Delta H(\boldsymbol{r},t')\right)dt' & = & \delta(\boldsymbol{r})\\
\Leftrightarrow & \int_{0}^{t}\delta(\boldsymbol{r})\delta(t')dt' & = & \delta(\boldsymbol{r})
\end{alignat*}

which is true. 

A solution for the Green's function using $\mathcal{L}=\partial_{t}-2D\Delta$
in 2 dimensions is $H(r,t)=\frac{1}{4\pi2Dt}\exp(\frac{-r^{2}}{4\times2Dt})$. 

$G(r,t)$ can then be computed:

\begin{equation}
G(r,t)=2\lambda C\left[\frac{E1\left(\frac{r^{2}}{8Dt'}\right)}{8D\pi}\right]_{0}^{t}\label{eq:G_r_t}
\end{equation}

where $E1$ is the exponential integral. Using $G(r,0)=C^{2}$ and
$\lim_{x\rightarrow+\infty}E1=0$ in Eq. (\ref{eq:G_r_t}), 

\begin{equation}
G(r,t)=2\lambda C\frac{E1\left(\frac{r^{2}}{8Dt}\right)}{8D\pi}+C^{2}
\end{equation}

\begin{equation}
\Leftrightarrow g(r,t)=\frac{2\lambda}{C}\frac{E1\left(\frac{r^{2}}{8Dt}\right)}{8D\pi}+1
\end{equation}

\section*{Results}

\section*{Discussion}


