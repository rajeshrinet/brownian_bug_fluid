
\section*{Introduction}

In the present work, we attempt to replicate the results of Young et al. 2001 ``Reproductive pair correlations and the clustering of organisms'' \cite{young_reproductive_2001}, which provides an analysis of the clustering of organisms in an homogeneous environment, quantified by the pair correlation function. Using an individual based model of independent, random walking particles (also called ``brownian bugs''), they show that coupling simple ecological processes such as birth and death with a basic description of a turbulent and viscous flow leads to the formation of elongated aggregates and therefore departure from the usual, homogeneous solution of the advection-diffusion-reaction equation for a large population. The failure of averaging at fine time and spatial scales could explain certain features of, e.g., planktonic communities.\\

Our first interest in this paper was mostly driven by a major ecological wonder: the so-called paradox of the plankton (Hutchinson et al. 1961), i.e. the surprising diversity of phytoplankton species competing for the same resources in a seemingly homogeneous environment. Phytoplankton is notably characterized by its patchiness at macro to micro scales (Lovejoy 2001,  Pinel-Alloul & Ghadouani 2007). The spatial distribution of organisms, and more precisely their aggregation, could explain part of their coexistence, as spatial clustering can help reduce interspecific interactions \citep{font-munoz_advection_2017} and/or enforce niche differences \citep{barton_impact_2014}. Hydrodynamics alone cannot be made responsible for fine scale clustering: due to their size, microphytoplankton organisms experience a mostly viscous environment in a laminar shear fields, with random, but homogeneous changes in directions due to turbulence \citep{peters_effects_2000}. However, ecological phenomena as simple as growth and death, which occur at the phytoplankton scale, interact with larger hydrodynamics processes and can lead to aggregates. In this context, a better understanding of the interactions between demographic stochasticity and environmental fluctuations at fine scales could provide further explanation for the distribution and coexistence of these organisms in turbulent environments. \\

In this replication, we aim not only to replicate the main results of the paper, but also to clarify and develop the mathematical background behind the main equations governing the dynamics of the model. 
 
\section*{Methods}

\subsection*{Brownian bug model}
The brownian bug model is a discrete-time, individual-based model, here presented in its 2D formulation. Each particle is characterized by the vector of its Cartesian coordinates $\mathbf{x}=\begin{pmatrix} 
      x_1\\ 
      x_2 
\end{pmatrix}$ and its original position on the y axis at t=0 (a child particle inherits this attribute). Space is a $L\times L$ square with periodic boundary conditions. Each timestep, of duration $\tau$, is divided into three substeps: (1) demographic processes, (2) diffusion, and (3) advection. \\

(1) The first substep is a Galton-Watson branching process. Each organism has a fixed probability ($p$) of reproducing, dying ($q$), or remaining unchanged ($1-p-q$). When an individual reproduces, a new organism appears on top of the parent. In this paper, $p=q=0.5$.

(2) Diffusion is modeled as a brownian motion, i.e. $\mathbf{x'}_k(t)=\mathbf{x}(t)+\delta\mathbf{x}(t)$ where each component of $\delta\mathbf{x}(t)$ follows a Gaussian distribution $\Gaussian (0,\Delta)$ where $D=\frac{\Delta^2}{2\tau}$ is the diffusivity. 

(3) The turbulent flow governing advective stirring follows the Pierrehumbert random map.

\begin{eqnarray}
 x_1(t+\tau)&=&x'_1(t)+(U\tau/2)\cos[kx'_2(t)+\phi(t)]\\
 x_2(t+\tau)&=&x'_2(t)+(U\tau/2)\cos[kx_1(t+\tau)+\theta(t)]\\
 \label{eq:pierrehumbert}
 \end{eqnarray}

 where $\phi(t)$ and $\theta(t)$ are random phases uniformly distributed between 0 and $2\pi$, $k=2\pi/L$ and $U$ is the stretching parameter. \\
 
Unless otherwise specified, each simulation is initialized with $N_0=20,000$ particles uniformly distributed in a $1\times 1$ square and run for 1000 timesteps.
 
\subsection*{Relation with the advection-diffusion-reaction approximation} 
In continuous time, the distribution of particles in conditions similar as those described in the brownian bug model can be approximated by the advection-diffusion-reaction (ADR) equation. 

\begin{equation}
\frac{dC}{dt}=D\nabla^2 C+(\lambda-\mu)C
\label{eq:ADR}
\end{equation}

where $C$ is the concentration of particles, $\lambda$ is the growth rate ($\lambda=p/\tau$)  and $\mu$ is the death rate ($\mu=q/\tau$). When $\lambda=\mu$, the solution of eq. \ref{eq:ADR} is $C(\mathbf{x},t)=C_0$ where $C_0$ is the initial uniform concentration. To compare the theory to the actual distribution, the brownian bug model is run without the advection component ($U=0$).\\

To assess the effect of the turbulent motion, the model is also run without its demographic component, but with advection and diffusion.

\subsection*{Pair correlation function G(r,t)}

The pair correlation function (hereafter, pcf) $G(\mathbf{x}_i,\mathbf{x}_j,t)$ is defined so that $G(\mathbf{x}_i,\mathbf{x}_j,t)dA_1dA_2$ is the probability finding a pair of brownian bugs with one member in the area $dA_1$ around $\mathbf{x}_1$ and the other in the area $dA_2$ around $\mathbf{x}_2$. The radial density function $g(r,t)$ is defined as $G(\mathbf{x}_i,\mathbf{x}_j,t)=C^2g(r,t)$ with $r=|\mathbf{x}_i-\mathbf{x}_j|$. As the pair correlation disappear when $r\rightarrow\infty$, $g\rightarrow 1$.  

We should note here that these notations differ from usual notations (e.g. Illian et al. 2008): pcf may correspond to $g$ while $G$ is defined as the pair density. We will keep the notations of \cite{young_reproductive_2001} hereafter. 

\subsubsection*{Derivation of G(r,t)}

Finding back Eq. (2) in the original paper?\\	

\begin{equation}
\frac{\partial G}{\partial t}=2Dr^{1-d}\frac{\partial}{\partial r}\left(r^{d-1}\frac{\partial G}{\partial r}\right)+2(\lambda-\mu)G+\gamma r^{1-d}\frac{\partial}{\partial r}\left(r^{d+1}\frac{\partial G}{\partial r}\right)+2\lambda C\delta(\boldsymbol{x})\label{eq:eq_2_Young_total}
\end{equation}

where $\boldsymbol{x}$ is the position of the particle.

We will focus on the case $d=2$ and $\lambda=\mu$, which means Eq.
(\ref{eq:eq_2_Young_total}) can be reduced to

\begin{equation}
\frac{\partial G}{\partial t}=\frac{2D}{r}\frac{\partial}{\partial r}\left(r\frac{\partial G}{\partial r}\right)+\frac{\gamma}{r}\frac{\partial}{\partial r}\left(r^{3}\frac{\partial G}{\partial r}\right)+2\lambda C\delta(\boldsymbol{x})\label{eq:eq_2_Young_reduced}
\end{equation}

The value of $\gamma$ is computed from simulations (see SI.)

\subsubsection*{Analytical solution with advection}

In the presence of advection ($\gamma\neq0$), a steady-state solution
can be found. 

\begin{align}
  &  \frac{2D}{r}\frac{\partial}{\partial r}\left(r\frac{\partial G}{\partial r}\right)+\frac{\gamma}{r}\frac{\partial}{\partial r}\left(r^{3}\frac{\partial G}{\partial r}\right)+2\lambda C\delta(\boldsymbol{x})\nonumber & = & & 0 \\
\Leftrightarrow & 2\pi r\left(\frac{2D}{r}\frac{\partial}{\partial r}\left(r\frac{\partial G}{\partial r}\right)+\frac{\gamma}{r}\frac{\partial}{\partial r}\left(r^{3}\frac{\partial G}{\partial r}\right)+2\lambda C\delta(\boldsymbol{x})\right)\nonumber & = & & 0 \\
 \Leftrightarrow  & 2\pi\left(2D\frac{\partial}{\partial r}\left(r\frac{\partial G}{\partial r}\right)+\gamma\frac{\partial}{\partial r}\left(r^{3}\frac{\partial G}{\partial r}\right)\right)+2\pi r2\lambda C\delta(\boldsymbol{x}) & = & & 0\label{eq:steady_state}
\end{align}

We can then integrate Eq. (\ref{eq:steady_state}) over a small
area centered on a particle, with radius $\rho$. Let us first note
that

\begin{align}
& \int_{\mathbb{R}^{2}}\delta(\boldsymbol{x})d^{2}\boldsymbol{x} & & = & & 1\nonumber \\
\Leftrightarrow & \int_{0}^{2\pi}\int_{0}^{\rho}\delta(r')\delta(\theta)r'dr'd\theta & & = & & 1\nonumber \\
\Leftrightarrow & 2\pi\int_{0}^{\rho}\delta(\boldsymbol{x'})r'dr' & & = & & 1\label{eq:delta_integration}
\end{align}

Using Eq. (\ref{eq:steady_state}) and (\ref{eq:delta_integration}),
we can integrate between 0 and $\rho$, 

\begin{align}
 & & 0 & & = & & 2\pi\left(2D\rho\frac{\partial G}{\partial r}+\gamma\rho^{3}\frac{\partial G}{\partial r}\right)+2\lambda C\nonumber \\
\Leftrightarrow & & \frac{\partial G}{\partial r} & & = & & -\frac{1}{2\pi}\frac{2\lambda C}{2D\rho+\gamma\rho^{3}}\label{eq:deriv_G_r}
\end{align}

Eq. (\ref{eq:deriv_G_r}) can now be integrated between $\rho$ and $\infty$, knowing that $G(\infty)=C^{2}.$

\begin{equation}
 & C^{2}-G(\rho) & = & -\frac{1}{2\pi}{\displaystyle \int_{\rho}^{\infty}}\frac{2\lambda C}{2Dr+\gamma r^{3}}dr\label{eq:deriv_G_r_int1}
\end{equation}

Using the variable change $u=2Dr+\gamma r^{3}$, the integral is equivalent
to $\int\frac{u'}{u}du$.

\begin{align}
 & C^{2}-G(\rho) & = & & -\frac{\lambda C}{2\pi}\frac{1}{4D}[\log(\gamma)-\log(\frac{2D}{r^{2}}+\gamma)]\label{eq:deriv_G_rint2}\\
\Leftrightarrow & G(\rho) & = & & \frac{\lambda C}{8\pi D}\log\left(\frac{2D+\gamma r^{2}}{\gamma r^{2}}\right)+C^{2}\label{eq:G_rho}
\end{align}

Finally, the pair correlation function $g=G/C^{2}$ is defined as

\begin{equation}
g=\frac{\lambda}{8\pi DC}\log\left(\frac{2D+\gamma r^{2}}{\gamma r^{2}}\right)+1
\end{equation}


\subsubsection*{Analytical solution without advection}

When $U=0$, $\gamma=0$ and there is no steady solution. We can get
back to Eq. (\ref{eq:eq_2_Young_reduced}). 

\begin{equation}
\frac{\partial G}{\partial t}=\frac{2D}{r}\frac{\partial}{\partial r}\left(r\frac{\partial G}{\partial r}\right)+2\lambda C\delta(\boldsymbol{x})\label{eq:g_without_advection}
\end{equation}

Assuming an isotropic environment (and switching to the polar coordinate system), this means

\begin{equation}
\frac{\partial G}{\partial t}-2D\Delta G=2\lambda C\delta(\boldsymbol{x})
\end{equation}

where $\Delta=\nabla^{2}$ is the Laplacian operator. \\

We therefore have 

\begin{equation}
\mathcal{L}G(\boldsymbol{x},t)=2\lambda C\delta(\boldsymbol{x})\label{eq:LG_lambda}
\end{equation}

where $\mathcal{L}$ is the linear differential operator $\partial_{t}-2D\Delta$. \\

We can use the Green's function H, defined with $\mathcal{L}H=\delta(\boldsymbol{x},t)=\delta(\boldsymbol{x})\delta(t)$. \\

By definition, we know that $G(y)=\int H(y,s)2\lambda C\delta(s)ds$
(where $y=(\boldsymbol{x},t)$) is a solution to Eq.(\ref{eq:LG_lambda}).

\begin{align}
 & G(\boldsymbol{x},t) & = & & 2\lambda C\int_{\mathbb{R}^{2}}\int_{0}^{t}H(\boldsymbol{x}-\boldsymbol{x}',t')\delta(\boldsymbol{x}')d^2\boldsymbol{x}'dt'\nonumber \\
\Leftrightarrow &  & = &  & 2\lambda C\int_{0}^{t}H(\boldsymbol{x},t')dt'\label{eq:g_int_H}
\end{align}

Eq.(\ref{eq:g_int_H}) can be used in Eq. (\ref{eq:g_without_advection}):

\begin{alignat*}{2}
 & \frac{\partial}{\partial t}\left(2\lambda C\int_{0}^{t}H(\boldsymbol{x},t')dt'\right) & & = & & 2D2\lambda C\Delta\int_{0}^{t}H(\boldsymbol{x},t')dt'+2\lambda C\delta(\boldsymbol{x})\\
\Leftrightarrow & \int_{0}^{t}\left(\frac{\partial H(\boldsymbol{x},t')}{\partial t'}-2D\Delta H(\boldsymbol{x},t')\right)dt' & & = & & \delta(\boldsymbol{x})\\
\Leftrightarrow & \int_{0}^{t}\delta(\boldsymbol{x})\delta(t')dt' & & = & & \delta(\boldsymbol{x})
\end{alignat*}

which is true. \\

A solution for the Green's function using $\mathcal{L}=\partial_{t}-2D\Delta$
in 2 dimensions is $$H(r,t)=\frac{1}{4\pi2Dt}\exp(\frac{-r^{2}}{4\times2Dt})$$. 

$G(r,t)$ can then be computed:

\begin{equation}
G(r,t)=2\lambda C\left[\frac{E1\left(\frac{r^{2}}{8Dt'}\right)}{8D\pi}\right]_{0}^{t}\label{eq:G_r_t}
\end{equation}

where $E1$ is the exponential integral. Using $G(r,0)=C^{2}$ and
$\lim_{x\rightarrow+\infty}E1=0$ in Eq. (\ref{eq:G_r_t}), 

\begin{equation}
G(r,t)=2\lambda C\frac{E1\left(\frac{r^{2}}{8Dt}\right)}{8D\pi}+C^{2}
\end{equation}

\begin{equation}
\Leftrightarrow g(r,t)=\frac{2\lambda}{C}\frac{E1\left(\frac{r^{2}}{8Dt}\right)}{8D\pi}+1
\end{equation}

\section*{Results}

We were able to reproduce the spatial distributions of brownian bugs in each case described in \cite{young_reproductive_2001}.\\

We can see in Fig. \ref{fig:spatial_fig1} the clumping of organisms which illustrates the failure of the ADR approximation at fine scales due to demographic processes. Fig. \ref{fig:spatial_fig2} also show that hydrodynamics alone cannot ensure cluster formation.  

\begin{figure}[H]
\begin{center} 
 \includegraphics[width=0.49\textwidth]{../code/figure/spatial_distribution_Fig1.png}
  \caption{Distribution of brownian bugs at different times in a simulation with $\Delta=10^{-3}$ and $U=0$: initial conditions with a Poisson spatial distribution (a), $t=100\tau$ (b) and $t=1000\tau$ (c). Each particle is identified by a color which corresponds to the initial position on the y axis of its ancestor at t=0.}
  \label{fig:spatial_fig1}
\end{center}
  \end{figure}

\begin{figure}[H]
\begin{center}
\includegraphics[width=0.49\textwidth]{../code/figure/spatial_distribution_Fig2.png}
  \caption{Distribution of brownian bugs with different processes in a simulation with $\Delta=10^{-3}$ and $U\tau/2=0.1$: without demographic processes at $t=30\tau$ (a), and with demographic processes $t=1000\tau$ (b). Each particle is identified by a color which corresponds to the initial position on the y axis of its ancestor at t=0.}
  \label{fig:spatial_fig2}
\end{center}
  \end{figure}
  
Fig. \ref{fig:pcf_Fig3} proved much more challenging. Retrieving the analytical solutions of eq. \ref{eq:eq_2_Young_reduced} was the most difficult. However, we also encountered issues when computing the pair correlation functions on simulations: for large values of $r/\Delta$, the pcf was set to 0 due to sampling issue. With the same initial conditions as the previous simulations, i.e.  $N_0=20,000$ particles uniformly distributed in a $1\times 1$ square, there were too many missing values to assess the fit of the simulation to the theory. We therefore chose to increase the number of particles to $N_0=200,000$ over an area of 10. We can see on Fig. \ref{fig:pcf_Fig3} that there are still missng values but that we can confirm that simulated and analytical pcf match. 

\begin{figure}[H]
\begin{center}
 \includegraphics[width=0.99\textwidth]{../code/figure/pcf_per_Utot_dx10m8.pdf}
 \caption{Logarithmic (a) and linear (b) plots of $g(r,t)$ versus $r/\Delta$, with $\Delta=10^{-7}$ and $U\tau/2=0,0.1,0.5,2.5$ at $t=1000\tau$. Solide lines result from simulations, dotted lines correspond to analytical solutions and the solid grey line indicates the $r^{-2}$ scaling predicted by eq. \ref{eq:eq_2_Young_total}.}
  \label{fig:pcf_Fig3}
\end{center}
  \end{figure} 
 
\section*{Discussion}

%Stuff that we could discuss: scales? Pierrehumber formulation for turbulence?

\section*{Acknoweldgements}
We are very grateful to William Young for his help through the theoretical part of Young et al. 2001. 
